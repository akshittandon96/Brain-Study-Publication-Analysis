\documentclass[12pt,letterpaper]{article}
\usepackage{fullpage}
\usepackage[top=2cm, bottom=4.5cm, left=2.5cm, right=2.5cm]{geometry}
\usepackage{amsmath,amsthm,amsfonts,amssymb,amscd}
\usepackage{lastpage}
\usepackage{enumerate}
\usepackage{fancyhdr}
\usepackage{mathrsfs}
\usepackage{xcolor}
\usepackage{graphicx}
\usepackage{listings}
\usepackage{hyperref}

\lstset{frame=lrb,xleftmargin=\fboxsep,xrightmargin=-\fboxsep}
\hypersetup{%
  colorlinks=true,
  linkcolor=blue,
  linkbordercolor={0 0 1}
}
 
\renewcommand\lstlistingname{Output}
\renewcommand\lstlistlistingname{Algorithms}
\def\lstlistingautorefname{Alg.}

\lstdefinestyle{Python}{
    language        = R,
    frame           = lines, 
    basicstyle      = \footnotesize,
    keywordstyle    = \color{blue},
    stringstyle     = \color{green},
    commentstyle    = \color{red}\ttfamily
}

\setlength{\parindent}{0.0in}
\setlength{\parskip}{0.05in}

% Edit these as appropriate
\newcommand\course{Statistical Methods in Research}
\newcommand\hwnumber{2}                  

\newcommand\NetIDb{Akshit Tandon - 1792038}      

\pagestyle{fancyplain}
\headheight 35pt
\lhead{\NetIDa}
\lhead{\NetIDa\\\NetIDb}                 % <-- Comment this line out for problem sets (make sure you are person #1)
\chead{\textbf{\Large Assignment \hwnumber}}
\rhead{\course}
\lfoot{}
\cfoot{}
\rfoot{\small\thepage}
\headsep 1.5em
\begin{document}
{\Large {\textbf{Question 1}}}

H0 : Variance = 0.01\\
H1: Variance $>$ 0.01\\
It is a normal distribution, we will divide assumed standard deviation of 0.2/2 = 0.1. Hence, variance is 0.01.\\

Since the computed statistics of 92.16  is greater than
chi square value of 23.68, the null hypothesis is 
rejected.

\begin{lstlisting}[label=R Code,caption=Q1 R Code Output]
> data_q1 <- read.csv("Q1.csv")
> given_sd = 0.1
> given_variance = given_sd * given_sd
> given_variance
[1] 0.01
> sample_var <- var(data_q1$ï..gain_loss)
> SS <- sample_var*14
> X_2 = SS/given_variance
> X_2
[1] 92.16933
> df = 14
> chi_sq = qchisq(0.95,df)
> chi_sq
[1] 23.68479
\end{lstlisting}
{\Large {\textbf{Question 2}}}\\

$H0$ : Delta  =  0\\
$H1$: Mean New $>$ Mean Standard 

P value of 0.98  $>$  0.05,  we do not reject the NULL 
hypothesis.
Hence, there is not enough evidence to claim that the new method is better than the standard method.

\begin{lstlisting}[label=R Code,caption=Q2 R Code Output]
      
> teaching <- read.csv("teachers.csv")
> t.test(teaching$New,teaching$Standard,paired = F,
+        alternative = "greater")

	Welch Two Sample t-test

data:  teaching$New and teaching$Standard
t = -2.3066, df = 28.839, p-value = 0.9858
alternative hypothesis: true difference in means is greater than 0
95 percent confidence interval:
 -5.047475       Inf
sample estimates:
mean of x mean of y 
 18.54375  21.45000 
\end{lstlisting}
{\Large \textbf{Question 3}}\\
{a)}\\
H0 : Delta = 0\\
H1: Mean New \neq Mean Standard \\

Pvalue $<$ 0.05, we do reject the NULL hypothesis. There is enough evidence that the new technology is better than the standard technology.\\
\begin{lstlisting}[label=R Code,caption=Q3(a) R Code Output]
      
> p1 = 74/150
> p2 = 73/91
> x1 = 74
> x2 = 73
> n1 = 150
> n2 = 91
> alpha = 0.05
> p_hat = (x1+x2)/(n1+n2)
> z_val = (p1-p2)/sqrt(p_hat*(1-p_hat)*(1/n1+1/n2))
> z_val
[1] -4.765629
> pnorm(z_val)
[1] 9.413283e-07
\end{lstlisting}

{b)}\\
H0 : Delta = 0\\
H1: Mean Age New \neq Mean Standard \\

pvalue $<$ 0.05, we reject the NULL hypothesis, hence the ages in the groups are not same. 


\begin{lstlisting}[label=R Code,caption=Q3(a) R Code Output]
 > s1_age_mean = 5.73
> s1_size = 150
> s2_age_mean = 9.02
> s2_size = 91
> SE <- sqrt((6.15^2/s1_size)+(6.10^2/s2_size))
> t <- (5.73-9.02)/SE
> t
[1] -4.046489
> df = s1_size + s2_size - 2
> pt(t,df)
[1] 3.511523e-05

\end{lstlisting}

{c)}\\
We got the better results using new technology probably because the children in the new technology group were older than the ones in the standard technology. \\

{\Large \textbf{Question 4}}\\
H0: ud = 0\\
H1: ud  $<$  0\\
This is a paired test. \\

Critical Value for degree of freedom = 7 and alpha = 0.01 = -3.0\\
t stat value of -2.3407 $>$ -3.00 , we cannot reject the NULL hypothesis.\\
Also, p value of 0.0259 $>$ 0.01, we do not reject the NULL hypothesis.\\

There is not enough evidence to conclude that the treatment increased the mean number of surviving fish. \\

\begin{lstlisting}[label=R Code,caption=Q4 R Code Output]
 > Untreated <- c(5,1,1.8,1,3.6,5,2.6,1)
> treated <- c(5,5,1.2,4.8,5,5,4.4,2)
> alpha = 0.01
> critical_t_val = -3.00
> t.test(Untreated,treated,paired = TRUE,conf.level = 0.99,
alternative = "less")

	Paired t-test

data:  Untreated and treated
t = -2.3407, df = 7, p-value = 0.0259
alternative hypothesis: true difference in means is less than 0
99 percent confidence interval:
      -Inf 0.4001223
sample estimates:
mean of the differences 
                 -1.425 

\end{lstlisting}
\\
{\Large \textbf{Question 5}} \\
Assuming normal distribution\\

H0 : Mean = 0.10\\
H1: Mean $<$ 0.10 \\

pvalue of 0.99 $>$  0.05, hence we do not reject the NULL hypothesis. There is not enough evidence to conclude
that less than 10\% of the customers drink another brand. 
\begin{lstlisting}[label=R Code,caption=Q5 R Code Output]
> p = 0.10
> p_hat = 18/100
> n = 100
> SD <- sqrt(p*(1-p)/n)
> SD
[1] 0.03
> z_score = (p_hat - p)/SD
> z_score
[1] 2.666667
> ans_a = pnorm(z_score)
> ans_a
[1] 0.9961696

\end{lstlisting}






\end{lstlisting}
\end{document}